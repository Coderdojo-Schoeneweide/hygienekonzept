\documentclass{coderdojoschoeneweide}
\usepackage[utf8]{inputenc}
\usepackage[T1]{fontenc}

\title{Hygienekonzept für die Durchführung von Workshops für Kinder und Jugendliche im CoderDojo Schöneweide (CoderDojo Deutschland e.V.)}

\begin{document}
	\maketitle

	\begin{Form}

		\section*{Grundlage}

		Dieses Hygienekonzept basiert auf der \href{https://www.berlin.de/corona/massnahmen/verordnung/}{SARS-CoV-2-Basisschutzmaßnahmenverordnung} vom 13. Dezember 2022 und den \href{https://ljrberlin.de/corona-jugendarbeit-berlin}{Empfehlungen für ein Hygienekonzept für die Jugendverbandsarbeit des Landesjugendrings Berlin} vom 7. Februar 2022.

		\section*{Gültigkeit}

        Sollte die Einrichtung, die den Veranstaltungsraum / das Veranstaltungsgelände zur Verfügung stellt, ein eigenes Schutz- und Hygienekonzept haben und dieses der Workshopleitung im Vorhinein zur Verfügung gestellt haben, sind die dort festgehaltenen  Maßnahmen und Regelungen auch für die Durchführung dieses Workshops zu berücksichtigen. Es gilt daher auch der Pandemieplan der HTW Berlin. Die Hochschule befindet sich momentan im "Öffnungsbetrieb", wodurch es keine Einschränkungen gibt. 

		\section*{Maßnahmen}

		\begin{itemize}
             \item[$\blacksquare$] \textbf{Krankheitssymptome} Personen mit grippeähnlichen Krankheitssymptomen dürfen nicht an der Veranstaltung teilnehmen.
  
			\item[$\blacksquare$] \textbf{Belüftung} Sollte die Veranstaltung in einem geschlossenen Raum stattfinden, sind die Mentor:innen dafür verantwortlich, dass dieser Raum ausreichend gelüftet wird.

			\item[$\blacksquare$] \textbf{Reinigung des Material und der Einrichtung} Mentor:innen stellen Desinfektionsspray bereit, sodass Teilnehmende dieses bei Bedarf nutzen können, um geteiltes Material oder den Arbeitsplatz zu reinigen.

           \item[$\blacksquare$] \textbf{Kontaktnachverfolgung} Sollte im Nachhinein eine bei einer Veranstaltung anwesende Person positiv auf das Corona-Virus getestet werden sodass der Verdacht besteht, dass sie bei der Veranstaltung Personen angesteckt hat, so wird darum gebeten, diese Information der Veranstaltungsleitung (schoeneweide.berlin@coderdojo.com) mitzuteilen, sodass diese die Information anonymisiert an die anderen Teilnehmenden weiterleiten kann.

            \item[$\blacksquare$] \textbf{Hust- und Niesetiquette} Anwesende werden gebeten bei Bedarf in die Armbeuge zu husten doer zu niesen .

            \item[$\blacksquare$] \textbf{Händewaschen} Anwesende werden gebeten, sich vor der Veranstaltung die Hände zu waschen. Desinfektionsmittel zur Desinfektion der Hände steht bereit.

            \item[$\blacksquare$] \textbf{Mund-Nasen-Bedeckung} Die Maskenpflicht entfällt. Es wird jedoch empfohlen, dass Teilnehmende eine FFP2 Maske tragen, um einer möglichen Übertragung durch Aerosole vorzubeugen. 

            Mentor:innen und Teilnehmende sind dazu angehalten, ihre eigenen medizinischen, bzw. FFP2 Masken mitzubringen.

           \item[$\blacksquare$] \textbf{Testpflicht} Die Testpflicht entfällt.

           \item[$\blacksquare$] \textbf{Beschränkung der Teilnehmendenanzahl} Die Beschränkung der Teilnehmendenanzahl entfällt.

           \item[$\blacksquare$] \textbf{Anwesenheitsdokumentation} Die Anwesenheitsdokumentation entfällt.

           \item[$\blacksquare$] \textbf{Beschränkung von Essen und Trinken } Die Beschränkungen zu Essen und Trinken entfallen.

           \item[$\blacksquare$] \textbf{Feste Sitzplätze} Die Beschränkung zu festen Sitzplätzen entfällt.

           \item[$\blacksquare$] \textbf{Vermeidung von Ansammlungen} Die Beschränkung zur Vermeidung von Ansammlungen entfällt.
		\end{itemize}

	\end{Form}
\end{document}
